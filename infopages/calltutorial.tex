\ifdefined\COMPLETE
\else
	\documentclass[10pt,twoside]{extarticle}

\usepackage{xcolor}

\usepackage{graphicx}

\usepackage{multicol}

\usepackage{background}

\usepackage{adjustbox}

\usepackage{ifoddpage}

%\usepackage{showframe}

\usepackage{fontspec}
\setmainfont{DejaVu Sans}
\setlength{\lefthyphenmin}{8}
\setlength{\righthyphenmin}{8}

\ifdefined\bleed
	\usepackage[paperheight=216mm, paperwidth=151mm, top=2.25cm, bottom=2.8cm, outer=1.1cm, inner=0.8cm,heightrounded,marginparwidth=0cm,marginparsep=0cm]{geometry}
\else
	\usepackage[paperheight=210mm, paperwidth=148mm, top=1.95cm, bottom=2.5cm, outer=0.8cm, inner=0.8cm,heightrounded,marginparwidth=0cm,marginparsep=0cm]{geometry}
\fi

\newcommand{\getdefaultbggraphic}{
	\ifdefined\nobackground
	\else
		\checkoddpage
		\ifdefined\bleed
			\ifoddpage
		   		\includegraphics[trim = 0 0 0 0,clip,width = \paperwidth,height = \paperheight, keepaspectratio]{images/songbackground_right.pdf}
			\else
		   		\includegraphics[trim = 0 0 0 0,clip,width = \paperwidth,height = \paperheight, keepaspectratio]{images/songbackground_left.pdf}
			\fi
		\else
			\ifoddpage
		   		\includegraphics[trim = 0 3mm 3mm 3mm,clip,width = \paperwidth,height = \paperheight, keepaspectratio]{images/songbackground_right.pdf}
			\else
		   		\includegraphics[trim = 3mm 3mm 0 3mm,clip,width = \paperwidth,height = \paperheight, keepaspectratio]{images/songbackground_left.pdf}
			\fi
		\fi
	\fi
}

\setlength{\parskip}{0em} 
\setlength{\parindent}{0em}
\setlength{\columnsep}{0.3cm}
\setlength{\columnseprule}{0.5pt}

\usepackage{fancyhdr}
\fancypagestyle{song} {
	\fancyhead[L]{\vspace*{1em}\makesongheader}
	\fancyhead[R]{\makesongcredits}
	\fancyfoot{}
	\fancyfoot[C]{\makebox[\textwidth][c]{--- {\thepage} ---}}
	\fancyhfoffset[r]{\dimexpr+0.2cm\relax}
	\renewcommand{\headrulewidth}{0pt}
	\renewcommand{\footrulewidth}{0pt}
	\setlength{\headheight}{35pt}
	\backgroundsetup{opacity = 1, scale = 1, angle = 0,
	   contents = {\getdefaultbggraphic}}
}
\fancypagestyle{default} {
	\fancyhead{}
	\fancyhead[L]{\ifdefined\pagetitletext\vspace*{1em}\pagetitle{\pagetitletext}\\\fi}
	\fancyfoot{}
	\fancyfoot[C]{\makebox[\textwidth][c]{--- {\thepage} ---}}
	\renewcommand{\headrulewidth}{0pt}
	\renewcommand{\footrulewidth}{0pt}
	\setlength{\headheight}{35pt}
	\backgroundsetup{opacity = 1, scale = 1, angle = 0,
	   contents = {\getdefaultbggraphic}}
}
\fancypagestyle{nopagenum} {
	\fancyhead{}
	\fancyfoot{}
	\renewcommand{\headrulewidth}{0pt}
	\renewcommand{\footrulewidth}{0pt}
	\setlength{\headheight}{35pt} 
}
\fancypagestyle{empty} {
	\fancyhead{}
	\fancyfoot{}
	\renewcommand{\headrulewidth}{0pt}
	\renewcommand{\footrulewidth}{0pt}
	\setlength{\headheight}{0pt} 
	\backgroundsetup{contents = {}}
}
\pagestyle{default}

\title{Aqours 3rd Live WONDERFUL STORIES - Unofficial Information Booklet}
\author{Suyooo}
\date{July 15, 2018}

\newcommand{\fullimagepage}[2]{
	\newpage
	\thispagestyle{nopagenum}
	\ifdefined\nobackground
		\backgroundsetup{contents = {}}
		\vspace*{\fill}
		\vspace{-5em}
		\begin{center}#2\end{center}
		\vspace{\fill}
	\else
		\backgroundsetup{opacity = 1, scale = 1, angle = 0,
			contents = {
				\ifdefined\bleed
					\includegraphics[trim = 0 0 0 0,
					width = \paperwidth,
					height = \paperheight, keepaspectratio]
					{images/#1.png}
				\else
					\includegraphics[trim = 0 3mm 3mm 3mm,
					width = \paperwidth,
					height = \paperheight, keepaspectratio]
					{images/#1.png}
				\fi
			}}
		~\vfill
	\fi
	\newpage
	\backgroundsetup{opacity = 1, scale = 1, angle = 0,
	   contents = {\getdefaultbggraphic}}
}

\definecolor{chika}{HTML}{F0A20B}
\definecolor{riko}{HTML}{E9A9E8}
\definecolor{kanan}{HTML}{13E8AE}
\definecolor{dia}{HTML}{F23B4C}
\definecolor{you}{HTML}{49B9F9}
\definecolor{yohane}{HTML}{898989}
\definecolor{hanamaru}{HTML}{E6D617}
\definecolor{mari}{HTML}{AE58EB}
\definecolor{ruby}{HTML}{FB75E4}
\definecolor{aqours}{HTML}{3299E9}
\definecolor{sarah}{HTML}{3BB7EB}
\definecolor{leah}{HTML}{98ADD8}

\definecolor{clap}{HTML}{00CC22}
\definecolor{special}{HTML}{DDAA00}

\usepackage{shadowtext}
\shadowoffset{2pt}
\shadowcolor{black!5!white}

% ALTERNATING LETTER COLORS from https://tex.stackexchange.com/a/285974
\ExplSyntaxOn
\NewDocumentCommand{\colorstring}{O{aqours!90!white,aqours!95!black}+m}{%
  \clist_set:Nn \l_tmpa_clist {#1}%
  \int_zero:N \l_tmpa_int%
  \str_set:Nx \l_tmpa_str {#2}%
  \int_step_inline:nnnn {1} {1} {\str_count:N \l_tmpa_str } {%
    \str_case_x:nnF {\str_item:Nn \l_tmpa_str {##1}} {%
      {\space}{\space}
    }{%
      \int_compare:nNnTF {\l_tmpa_int } < {\clist_count:N \l_tmpa_clist } {
        \int_incr:N \l_tmpa_int
      }{%
        \int_set:Nn \l_tmpa_int {\c_one}
      }
      \textcolor{\clist_item:Nn \l_tmpa_clist {\l_tmpa_int }}{\hspace{-0.43em}\shadowtext{\str_item:Nn \l_tmpa_str {##1}}}
    }
  }
}
\ExplSyntaxOff

\newcommand{\call}[1]{{\textbf{\footnotesize{\textcolor{red}{(#1)}}}}}
\newcommand{\calltext}[1]{{\textbf{\textcolor{red}{#1}}}}
\newcommand{\callside}[1]{\hfill\call{#1}}
\newcommand{\callline}[1]{\vspace{-0.83em}\call{#1}\vspace{0.5em}}
\newcommand{\clap}[1]{{\textbf{\footnotesize{\textcolor{clap}{(#1)}}}}}
\newcommand{\clapline}[1]{\vspace{-0.83em}\clap{#1}\vspace{0.5em}}
\newcommand{\claptext}[1]{{\textbf{\textcolor{clap}{#1}}}}
\newcommand{\specinst}[1]{{\textbf{\footnotesize{\textcolor{special}{(#1)}}}}}
\newcommand{\specinstline}[1]{\vspace{-0.83em}\clap{#1}\vspace{0.5em}}
\newcommand{\specinsttext}[1]{{\textbf{\textcolor{special}{#1}}}}
\newcommand{\sing}[1]{{\textbf{\textcolor{aqours}{#1}}}}
\newcommand{\memberbg}[2]{\setlength{\fboxsep}{0pt}\colorbox{#1!30}{\makebox[\linewidth][l]{\strut{#2}}}}
\newcommand{\memberbgnote}[3]{{\footnotesize{\textcolor{#1}{\textbf{#3}}}}\\\memberbg{#1}{#2}}
\newcommand{\memberbgside}[3]{\memberbg{#1}{\strut{#2\hfill{\footnotesize{\textcolor{#1}{\textbf{#3}}}}}}}


\newcommand{\pagetitle}[1]{%
	\shadowcolor{aqours!25!white}%
	{\LARGE\textbf{\strut{\colorstring{#1}}}}%
	\shadowcolor{black!5!white}%
}

\newcommand{\cancel}[2]{%
    \tikz[baseline=(tocancel.base)]{
        \node[inner sep=0pt,outer sep=0pt] (tocancel) {#1};
        \node[red] at (0,0.66) {#2};
        \draw[line width=1mm, red] (tocancel.south west) -- (tocancel.north east);
    }%
}%

\def\changemargin#1#2{\list{}{\leftmargin#1\rightmargin#2}\item[]}
\let\endchangemargin=\endlist

\newcommand{\makesongheader}{
	\pagetitle{\songtitle}\\
	{\small\color{black!40!aqours}{\textbf{\songcomment}}}
}
\newcommand{\makesongcredits}{
	\adjustbox{scale={0.85}{1}}{\scriptsize{
		\color{black!40!aqours}{Writer \textbf{\songwriter} Lyrics \textbf{\songlyrics} Arrangement \textbf{\songarrange}}
	}}
}

\usepackage[hidelinks]{hyperref}

\ifdefined\COMPLETE
	\graphicspath{{./}}
\else
	\graphicspath{{../}}
\fi

	\begin{document}
\fi

\def\pagetitletext{What are Calls?}

Audience participation is an important part of idol concerts and Love Live is no exception. During certain parts of a song, fans will perform coordinated calls and light stick motions, known as \textbf{"wota"}. Sometimes, the artists themselves will join as well, signaling when to start certain calls or chants.\\

Below is a list of the most common calls in Love Live songs. If you feel lost at any point, don’t worry - follow the people around you and go with the flow!\\
\vfill

You will see people performing calls with \textbf{penlights}, electrically powered LED lights that can glow in either a single color or cycle through multiple ones. They are also known as Kingblades, named after a popular brand of such electric lights.\\

You can buy penlights from various places online (Mandarake, FromJapan, Amazon, eBay...). There are also special penlights released for Aqours Lives, although they’re of limited quantity and more expensive than others.\\

Idols usually have \textbf{image colors}, of which Love Live! Sunshine!! is not exempt from. As such, waving penlights to the rhythm of the music, set to the image colors of your favorite idols, is the standard way of supporting them during their lives, while not being too obstructive.\\

While lightsticks (or snap lights) differ from penlights in that they generate light chemically and are generally dimmer in most cases, they still can be used if you don't own an LED light. You'll be limited to one color per stick, but you can still support the performers just the same.\\

\vfill

\textbf{Keep in mind that there is no “right way” to call!} Don’t worry if you occasionally chant at different times than other people, if it fits the beat and mood of the song you’re good to go!\\

\textbf{Doing calls is not obligatory!}  You can always enjoy the DV with no calls at all! But keep in mind that they are a huge part of the experience, so why not ask for a light from a fellow attendee and join the fun?

\vfill

\newpage
\vspace*{0.5em}
\textbf{\large \colorstring{Basic Calls}}\\

\textbf{Hai Chant}\\
Just yell "Hai!" to the beat. Swing your light forwards each time. This chant is usually used during upbeat instrumental parts of a song.\\

\textbf{Fu Chant}\\
"Fu" can be used in various ways, for example to highlight certain rhythms in a song or going along with a prominent instrument in the song. Sometimes the "u" is emphasized, indicated by "fu--" in this callbook.\\

\textbf{Kecha}\\
Not actually a call per se, but a "furi", a certain motion you perform.
Move your hand in front of you, slowly raise it up to point towards the performers, and at the top, draw it back to your body again. Just imagine you are drawing a circle to the beat, because that is basically the motion of this furi.\\
Used during slow songs or parts with no percussion.
\vfill
\textbf{\large \colorstring{Advanced Calls}}\\

\textbf{Seno Call}\\
"Seno" (meaning something like "ready, set...!") is used to set up another call/chant. Usually followed by a Hai Chant combo and/or PPPH (see below).\\

\textbf{PPPH}\\
A possible follow up to a Seno call by repeatedly chanting "ooooh-- hai!" during the pre-chorus. PPPH is usually followed by a Hai Chant.\\
(Bonus points if you bend your knees just slightly and do a little drum roll motion with your glowsticks during the “ooooh” part.)\\

\vfill

\ifdefined\COMPLETE
\else
	\end{document}
\fi